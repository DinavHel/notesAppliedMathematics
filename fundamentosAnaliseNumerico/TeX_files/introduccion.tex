\chapter*{Introducción}

Manuel Antonio e Álvaro Cebreiro asinaban, en 1922, o manifesto "Máis alá!",
que comezaba dicindo o seguinte:\\
{\it Sin pretensións de suficenza doutoral, nin de ningún outro xeito somellante, a rebeldía d´uns mozos galegos fai esta chamada a Mocedade intelectual d´a nosa Terra.}\\
Comezo eu, pois, este libro do mesmo xeito: sen pretensións de suficiencia doutoral,
a rebeldía deste estudante de matemáticas fai esta chamada á mocidade matemática
(e á mocidade non tan moza) da nosa Terra.

Quizais é simple descoñecemento pola miña parte, pero atopo a cantidade de textos
en galego sobre o campo da análise numérica e a matemática aplicada... deficiente.
O único libro que coñezo é a "Introducción á Álxebra Numérica Matricial e á
Optimización" de Philippe G. Ciarlet (libro que considero magnifico e cunha
traducción deliciosa), que foi escrito (se non me fallan os datos) nos 80
e traducido ó galego nos 90. Se ben os resultados que expón siguen sendo
válidos e merecedores de consulta, certas afirmacións (en especial o referido
ó estado actual da computación) non teñen sentido no día de hoxe. Máis de unha
vez atopeime sorprendido ante a mención de, por exemplo, rutinas de Algol,
unha linguaxe de programación que non vexo que se manteña demasiado vixente.
Para min resulta moi interesante, igual que disfruto moito lendo o REALP de
Wilkinson (Rounding Errors in Algebraic Processes): ler textos de hai uns anos
pon en contexto canto cambiaron as cousas no mundo dende os 60 ou mesmo dende os 80
ou 90. Pero un alumno de primeiro ou segundo de matemáticas da USC, non versado
na historia da computación, pode atoparse levemente confuso ou mesmo sorprendido
á hora de verse consultando textos máis vellos que el. Foi, en certo xeito, o meu
caso: tendo nacido no ano 1996, as primeiras edicións de varios dos meus libros
de referencia teñen máis anos ca min.